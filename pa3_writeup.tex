\documentclass[11pt]{article}
\title{CS 124 - Programming Assignment 3}
\author{Aidan Daly, Byron Hood}
\usepackage{fullpage}
\usepackage{graphicx}
\usepackage{epstopdf}

\begin{document}
\maketitle

\section{Dynamic Programming Solution to NumPartition}
If $n$ is the length of the sequence $S$ of numbers and $b$ is the sum of the sequence, define a boolean matrix $M$ such that $M(i,k)$ is true if there exists a non-empty subset of the first $i$ elements of$S$ that sums to $k$.  Store a parallel array $A$ that keeps track of the members of the subset in the corresponding cell of $M$.  Thus, the solution to our original problem would be the subset $A(n,j)$ such that $|j-\frac{b}{2}|$ is minimized and $M(n,j)$ is true.

We can initialize all $M$ and $A$ to $false$ and $\emptyset$, respectively.  Then, for all $1\leq j \leq b$, if $s_1 == j$ set $M(1,j) = true$ and $A(1,j) = s_1$.  Then, for all $2 \leq i \leq n$:
\begin{equation}
M(i,j) = M(i-1,j) || M(s_i == j) || M(i-1,j-s_i) \mbox{ for all } 1 \leq j \leq b
\end{equation}
Where if the first clause is true, we set $A(i,j) = A(i-1,j)$, if the second clause is true, we set $A(i,j) = \{s_i\}$, and if the third clause is true, we set $A(i,j) = A(i-1,j)+\{s_i\}$.

The tables $M$ and $A$ will be filled in $nb$ time.  Afterwards, minimizing $|j-\frac{b}{2}|$ such that $M(n,j) == true$ takes $b$ time.  Thus, the runtime of this DP solution is $O(nb+b) = O(nb)$, which is indeed pseudo-polynomial time.  After we find the $j$ that satisfies this, we simply return $A(n,j)$ as the minimum partition and $|b-j|$ as the residue.

\section{Efficient Implementation of Karmarkar-Karp Algorithm}
The Karmarkar-Karp algorithm relies on repeatedly removing the two largest elements from the sequence and inserting their difference.  Since this must run for $n$ iterations on a list of size $n$ (iterating through pairs and differencing them), we must make sure that only $log(n)$ work is done at each step.

A Priority Queue (or heap) implementation would achieve this.  First, build a priority queue of the elements in the sequence.  We know that constructing a heap of $n$ elements takes $O(n)$ time, so this setup step is linear.  Then, remove the largest two elements from the queue.  This will operate in $2log(n) = O(log(n)$ time.  Finally, insert their difference back into the queue ($O(log(n))$ time), and repeat until there is only one element remaining.  Return the last element. 

Since the setup takes linear time, and each point in the iteration (removing, differencing, inserting) takes $O(log(n))$ time, we have an $O(nlog(n))$ implementation of Karmarkar-Karp with a priority queue.

\section{Results of Local Search Algorithms}

\section{Karmarkar-Karp as starting point for local search algorithms}

\end{document}